L'intérêt majeur de la détection de contours est la réduction d'information. En effet, l'information d'une image peut être résumée par les contours des différents objets qu'elle contient car les contours contiennent les parties les plus informatives de l'image.

Pour certaines applications (reconnaissance de formes, imagerie médicale, cartographie,...), les contours de l'image suffisent pour l'exploiter. La détection de contours permet ainsi d'éviter de stocker des informations lourdes inutilement.

Dans ce rapport, nous allons nous intéresser au détecteur de contours de Canny-Deriche. \\

Le filtre de Deriche est un opérateur de détection de contours développé par Rachid Deriche en 1987. C'est un algorithme permettant la détection de contours dans une image à deux dimensions. Cet algorithme est une évolution du filtre de Canny développé par John Canny en 1986. 
